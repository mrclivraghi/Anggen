TESI
Introduction

context 

Every year hundreds thousands people gain access to the web by using all kind of devices like smartphone, tablet or desktop computer. In Europe and USA 80% of the people go online every day (source: Consumer Barometer by Google, 2015).

This breakneck increase has changed the equilibrium in the development world on behalf of web application, accessible from every device using a browser.
In this scenario building web application became crucial, while desktop software is losing slices of the market.

Development costs are quite high and a lot of time is wasted in repetitive and non-core actions that would be automized.

problem statement

Full stack developers have in charge the development of both server and client side of an application. The bootstrap of a project is not immediate because the developers must write hundreds line of standard and repetitive code that does not hit the core of the project.
The REST architecture defined a standard way to communicate between client and server applications. This imposes a rigid scheme to developers that build APIs similar to each other for every project, losing time and productivity in this simple tasks of copy and paste.
Even on the client side there is a waste of time in the coding of the standard forms for the various resources.
This problem is partially sovled by automatic generators that generate the standard code allowing the developer focusing on the business logic of the application.


But often automatic generators create source code that is complicated to understand and it cannot be easily modified, so developers often prefer to write their own code and copy-paste code from different APIs. The existing generators do not follow the most recent technology for the web applications and they can be hard to integrate with an already existing system.

Another statement of the problem is the documentation writing: it is often underestimated or ignored, in the best cases it is entrusted to low-level employees.
This brings to documentation that is not really helpful to future developers and does not show in detail the specific of the application.

In a RESTful system a complete set of documentation for the REST APIs is fundamental: client side developing can be totally decoupled and may be commit to a cheaper third part.


proposed solution

The proposed solution will generate a complete set of RESTful services, UI views and documentation.
The key idea of the proposed solution is simplicity: generated code must be easy to modify,understand and integrate in the current system. 
The generation is divided in two main modules: the REST APIs and the UI files.
These two modules are intended to be completely integrable with different systems. The UI generated module can be linked with an existing set of REST APIs using every technologies and the generated REST API’s completely stick the standard REST architecture.

The implementation proposed has been written in Java and use two different technologies for the server module and for the client module.
The server module is implemented on top of the Spring Framework, in the J2EE World. Among this it uses Spring Data (a framework to manage persistence operations) over JPA (standard Java Persistence Api), Jackson (a library used to serialize POJO to JSON and deserialize JSON to Java Objects),..
The generated REST APIs are documented with the Swagger library, that provides a complete description of every operation and the associated unit testing. In this scenario the REST APIs can be tested by using a simple user interface.

The UI module is generated over AngularJS framework. AngularJS is a relatively new framework supported by Google and Amazon that brings the MVC pattern to another level in the client side. Bootstrap3 has been used to stylish the UI.

The generation use the concept of annotation, a form of metadata that provide data about a program that is not part of the program itself but they do not provide effects on the operation of the code they annotate.

The crucial point of the generation is the domain model. The input of the generation process must be a set of Java classes modeling all the aspects of the problem and can be annotated with project-specific annotations.

In the first step the information contained in the domain model are interpreted and stored in a database. The information in the database can be easily modified even after the first step by using a UI interface generated by the generator itself.

The second step is the proper generation of the different modules. It takes the data from the database and create the source code of the different modules.

In addition to the standard files it generated other classes to manage the application configuration, the security of the APIs and the standard navbar for the UI interface.

Another aspect that has been implemented is the possibility to generate a small part of the project and to manage the modifies developed. This feature has been developed using GIT concepts at runtime.




structure of the thesis 
In the first chapter there are general information about the main concepts applied, deepeing the concepts of web application, REST architecture, MVC pattern and some details about the technologies adopted.
Then the state of art is analyzed in the chapter 3, discussing the main advantages and critical issues of the existing solution.
The idea is explained in the 4 chapter, focusing on the algorithm adopted for the generation and the main features of the framework.
In the third chapter there are technical details and issues about the implementation realized following a chronological order.
After that there is the analysis about testing the realised framework in two real world scenario, before ending with the conclusions.
The last chapters are dedicated to the content index and to bibliography.

